\section{CGCA on BlueCrystal}

I use phase 2. I normally use one 8-core
2.8GHz 8GB node, i.e. 1GB/core.
As of 10-MAR-2013 there is no cluster(?)
Intel license on BC, so it is not possible
to run coarray codes on multiple nodes, only
a single node.
I use \texttt{ifort} 12.0.2 20110112.

The timings below are from test \texttt{testAAL},
which is solidification
only, with 2 large writes of the whole super
array, one in the beginning and the other at
the end.
The grain resolution is $10^{-5}$.
Cubic array and coarray layout is used in all
runs, e.g. the coarray is declared as
\texttt{(n,n,n)[2,2,2]}, where $n$ is the
number of cells along one dimension of the
cube.
The timings are in Tab. \ref{tab:bc:test:aal}:

\begin{table}[h]
\centering
\begin{tabular}{cclllp{8em}}
\hline
\\
$n$	&grains	& mem/image	&node mem	&node used	&elapsed time \\ 
	&	& GB		&		&		& h:m:s \\
\hline
\\
400	&5120	& 0.5	&		&		&50m \\
450	&7290	&0.68	&8155144k	&7937716k	&1:10:24 \\
460	&7786	&0.73	&		&		&1:18:56 \\
470     &8305   &0.77   &8155148k       &8100948k       &1:22:55 \\
480     &8847   &0.82   &               &               &1:26:24 \\
490     &9411   &0.88   &               &               &1:29:59 \\
490     &9411   &0.88   &               &               &0:48:36, no array output \\
500	&10000	&0.93   &8155148k       &8108244k       &$>10$ hours, Swap: 7776616k used \\
\hline
\end{tabular}
\caption{Coarray runtimes for a single node on BlueCrystal phase 2.
Test \texttt{testAAL}.}
\label{tab:bc:test:aal}
\end{table}

Note: looking at 490 model, two file write times
can be as high as 40 min!

 {\bf Conclusion}: for this test, at least, the biggest
array of 4 byte integers seems to be $490^3$.
