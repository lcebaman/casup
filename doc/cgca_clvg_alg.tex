\section{Cleavage algorithm}

A grain (crystal) is described by its rotation
tensor, $\tenrtwo{R}^c$, with the usual meaning:
a vector in the crystal coord. system, $x^c$,
is transformed into a vector in spatial (cellular) coord. system,
$x^s$, as

\begin{equation}
 x^{s}
  =
   \tenrtwo{R}^{c} \cdot x^{c}
 \label{eq:cgca:pub}
\end{equation}

It is assumed that the cleavage is controlled by
$\sigma_1$, the max principal stress.
If $\sigma_1^s$ is the max principal stress vector
in the spatial coord. system, then
$\sigma_{1}^{c} = \tenrtwo{R}^{s} \cdot \sigma_{1}^{s}$
is the max principal stress vector in the crystal
coord. system.
Here:

\begin{equation}
 \tenrtwo{R}^{s}
 \equiv
 \tenrtwo{R}^{-c}
  \equiv
   \left (
           \tenrtwo{R}^{c}
   \right)^{-1}
    \equiv
     \left (
             \tenrtwo{R}^{c}
     \right)^{T}
\label{eq:cgca:wonk}
\end{equation}

Each crystal plane, \{hkl\}, has a partucular
surface energy, $\gamma_{hkl}$.
We postulate that the work of cleavage
is equal to the surface energy.
The work of cleavage is $\sigma_{hkl}$ times the distance necessary
to break the atomic bonds.
Following the ideas of \cite{gilman1959},
we take this distance equal to $a_0$, the
relaxation distance, which is the atom
diameter in the cleavage plane.
So the cleavage condition will be

\begin{equation}
 \sigma_{hkl}
   a_0
    =
     \gamma_{hkl}
\label{eq:cleav}
\end{equation}

from which the stress required to
cleave \{hkl\} plane is

\begin{equation}
 \sigma_{hkl}
  =
   \frac{ \gamma_{hkl}} { a_0}
\label{eq:muk}
\end{equation}

From \cite{gilman1959}, $\gamma_{100}$=1440,
$\gamma_{110}$=1710 and $\gamma_{111}$=5340erg/cm$^2$
(1erg/cm$^2$ = 10$^{-3}$J/m$^2$) and
$a_0=1.37 \times 10^{-10}$m.
This gives:
$\sigma_{100}=1.05 \times 10^4$MPa,
$\sigma_{110}=1.25 \times 10^4$MPa,
$\sigma_{111}=4.90 \times 10^4$MPa.
These values must be scaled, of course, because
the CA model is not on the atomic scale and because
material imperfections lower these stresses dramatically.
Anyway, the important factor is that we can differentiate
between different cleavage planes based on their
surface energies.

In BCC crystals there are 24 symmetric rotation tensors,
$\tenrtwo{R}_{sym}^{1\ldots 24}$,
including the identity tensor, 
So, if $n_{hkl}$ is a unit normal vector to some \{hkl\} plane,
then $n_{hkl}^{1\ldots 24} = 
\tenrtwo{R}_{sym}^{1\ldots 24} \cdot n_{hkl}$ are 24 normal
vectors describing all planes of the same class.

It is useful to split the maximum principal stress vector,
$\sigma_{1}^{c}$, into the magnitude, $||\sigma_{1}^{c}||$,
and the unit direction vector, $e_\sigma$:
$\sigma_{1}^{c} = ||\sigma_{1}^{c}|| e_\sigma$.

The maximum principal stress resolved to a \{hkl\} plane
is $S^{hkl} = ||\sigma_{1}^{c}|| e_\sigma \cdot n_{hkl}$.

Finally, because for cleavage analysis we are looking
for the weakest planes, we want to choose those planes
which maximise $S^{hkl}$, i.e.

\begin{equation}
 S^{hkl}
  =
   ||\sigma_{1}^{c}||
    \max
     \left|
       e_\sigma \cdot n_{hkl}^{1\ldots 24}
     \right|
\label{eq:hok}
\end{equation}
%
where the only planes under consideration
are \{100\}, \{110\} and \{111\}, although the surface
energy of \{111\} planes is so high that it is practically
impossible to cleave those.
Even if $e_\sigma \cdot n_{111}=1$, the resolved stress
on \{110\} plane will still be higher, and the cleavage
is likely to occur on \{110\}.
In addition to finding three maximum $S^{hkl}$ values,
one for each \{hkl\} orientation, we need to store
the three corresponding normal vectors, which
maximise the stress, $n_{hlk}^{max}$.

From \eqref{eq:muk} cleavage will occur when
$S^{hkl} \ge \gamma_{hlk}/a_0$.
For the algorithm it is useful to define:
$p^{100}=S^{100}/(\gamma^{100}/a_0)$,
$p^{110}=S^{110}/(\gamma^{110}/a_0)$,
$p^{111}=S^{111}/(\gamma^{111}/a_0)$
and 
$p_{max}=\max( p^{100}, p^{110}, p^{111})$.

The first part of the cleavage algorithm
can be thus summarised as follows:

\begin{algorithm}[H]
\SetAlgoLined
%\SetKwData{And}{and}
\SetKwInOut{Input}{input}
\SetKwInOut{Output}{output}

\Input{ $\sigma_{1}^s$, $\tenrtwo{R}^c$ }
\Output{ $p^{100}$, $p^{110}$, $p^{111}$, $p^{max}$,
 $n_{100}^{max}$, $n_{110}^{max}$, $n_{111}^{max}$ }
\BlankLine
$\sigma_{1}^{c} = (\tenrtwo{R}^c)^T \cdot \sigma_1^s$\;
$ S^{100} = ||\sigma_{1}^{c}||
  \max \left| e_\sigma \cdot n_{100}^{1\ldots 24}
     \right|$; $n_{100}^{max}$ \;
$ S^{110} = ||\sigma_{1}^{c}||
  \max \left| e_\sigma \cdot n_{110}^{1\ldots 24}
     \right|$; $n_{110}^{max}$ \;
$ S^{111} = ||\sigma_{1}^{c}||
  \max \left| e_\sigma \cdot n_{111}^{1\ldots 24}
     \right|$; $n_{111}^{max}$ \;
$p^{100}=S^{100}/(\gamma^{100}/a_0)$ \;
$p^{110}=S^{110}/(\gamma^{110}/a_0)$ \;
$p^{111}=S^{111}/(\gamma^{111}/a_0)$ \;
$p_{max}=\max( p^{100}, p^{110}, p^{111})$\;
\caption{
Cleavage algorithm, calculating max resolved
stress}
\label{algo:clvg1:p}
\end{algorithm}

There are special cell states representing
cleavage cracks on \{100\}, \{110\}, \{111\} planes,
$s_{100}$, $s_{110}$, $s_{111}$.
We call the set of all these states {\em cleavage states},
$s_c=[s_{100}, s_{110}, s_{111}]$.

Finally, we can decide whether cleavage will happen,
and if so, on which plane.
The algorithm can be constructed as follows.
The outputs are the unit vector, $n^s$, normal to the
active cleavage plane, in the spatial coord. system,
and the cleavage cell state, $s$.
The vector $n^c$ is first calculated in the grain
coord. sys. and then rotated to the spatial coord. sys.,
$n^s$.

\begin{algorithm}[H]
\SetAlgoLined
%\SetKwData{And}{and}
\SetKwInOut{Input}{input}
\SetKwInOut{Output}{output}

\Input{ $p^{100}$, $p^{110}$, $p^{111}$, $p_{max}$,
 $n_{100}^{max}$, $n_{110}^{max}$, $n_{111}^{max}$ }
\Output{ $n^s$, $s$, flag }
\BlankLine
{$n^s$=(0,0,0), $s=0$, flag=FALSE}\;
\If{$p_{max} \ge 1$}{
flag=TRUE\;
cleavage on \{100\}, $n^c=n_{100}^{max}$, $s=s_{100}$\;
\If{$p^{110} > p^{100}$}{
cleavage on \{110\}, $n^c=n_{110}^{max}$, $s=s_{110}$\;
}
\If{$p^{111} > p^{100}$ and
    $p^{111} > p^{110}$}{
cleavage on \{111\}, $n^c=n_{111}^{max}$, $s=s_{111}$\;
}
$n^s = \tenrtwo{R}^c \cdot n^c$\;
}
\caption{
Cleavage algorithm, calculating
the cleavage plane}
\label{algo:clvg1:n}
\end{algorithm}

So now we know the cleavage vector, $n$,
in the spatial coord. system
and the cleavage type (state), $s$.
This is all that is required to propagate a
type $s$ crack through the cellular automata.

\subsection{Cleavage representation in the cellular model}

Imagine that some cells with states
of the $s_c$ set are
randomly scattered across the model.
We call these {\em crack nuclei}.
The model is subjected to some max. principal
stress vector, $\sigma^1$.
The question we want answered is this:
how will the cracks grow from the nuclei?

One possibility is to analyse each 
crack edge cell.
Importantly, this immediately leads to the
need to have two coarrays, one for storing
grain numbers, and another for storing
fracture states of cells.
By analysing both arrays together one
can know what grain a given fractured
cell belongs.
As shown in Sec. \ref{sec:space:coarray},
we implement this by using a 4D coarray.

We scan over all undamaged cells.
If there is a cleaved neighbour,
such that the vector connecting the
cleaved and the central cells is on
or near the cleavage plane, then the
state of the central cell is changed
to the given cleavage state.
Note that it is possible that the given
cleavage state and the neighbour cleavage
state will differ.
The algorithm is summarised as follows:

\begin{algorithm}[H]
\SetAlgoLined
%\SetKwData{And}{and}
\SetKwInOut{Input}{input}
\SetKwInOut{Output}{output}

\Input{ $n^s$, $s$, threshold }
\Output{cell state change}
\BlankLine
 \For { i=1,26 } {
  \If { $i$th cell is cleaved } {
   $e_i$ is a unit vector from
    the central cell to $i$th cell\;
   \If { $|e_i \cdot n^s| <$ threshold } {
    central cell state is changed to $s$\;
    exit\;
   }
  }
 }
\caption{
Cleavage algorithm, propagating cleavage
crack through the cellular model.
}
\label{algo:clvg2}
\end{algorithm}

According to \cite{shterenlikht2013c},
if the threshold is 0.1733, then there will
always be at least 2 cells close enough
to the cleavage plane, so that they can
be considered cleaved.
However, this is probably unsatisfactory,
as the crack will propagate as just a straight
line in this case, not as a 2D plane.
So, perhaps the threshold must be lifted so
that there are at least 4 cells on the cleavage
plane in all cases.

The above algorithm will change state only of the
neighbouring cells.
Thus the speed of crack propagation is 1 cell/increment.

\subsection{Open questions}

Now the really hard bit: somebody must deside how often
to run the first part of the algorithm, i.e.
$(\sigma_1^s) \to (n^s,s)$.
In other words, some decision must be made regarding
how fast $\sigma_1^s$ might be changing, therefore
how often to recalculate the $n^s,s$ pair.

Equally crucial: how large an area of the complete
cellular model is subjected to the same $\sigma_1^s$?
Is it always the whole model?
Is it some small region of the complete cellular model?

These questions are the key for designing a parallel
algorithm.
Is each image using the same $\sigma_1^s$, or
separate?

For now I make a massive assumption that the complete
cellular model is always subjected to the same
macroscopic $\sigma_1^s$.
This means that all images use the same $\sigma_1^s$.

The cleavage algorithms are implemented as purely
serial routines.
Some parallel driver routines will have to go on top. 
A prototype algorithm, executed by each image, is as follows:

\begin{algorithm}[H]
\SetAlgoLined
%\SetKwData{And}{and}
\SetKwInOut{Input}{input}
\SetKwInOut{Output}{output}

\Input{cellular array, $\tenrtwo{R}^c$ array}
\Output{possible cell state change to cleaved}
\BlankLine
\texttt{old state} = liquid\_state\;
\For { all cells }{
pick cell i\;
read its grain state, \texttt{new state}\;
\If { grain of cell i is intact }{
 \If { \texttt{new state} $\ne$ \texttt{old state} }{
  run algs. \ref{algo:clvg1:p} and
   \ref{algo:clvg1:n}: $n^s$, $s$, flag\;
  \texttt{state old = state new} \; 
 }
 \If { flag = TRUE }{
  run alg. \ref{algo:clvg2}: change state
   of cell $i$ to $s$\;
 }
}
}
\caption{
Complete cleavage algorithm, top level view.
}
\label{algo:clvg:top}
\end{algorithm}

Note that the \texttt{new state} $\ne$
\texttt{old state} check
is intended for crossing the grain
boundary.
Because the grain orientation has changed,
the algs. \ref{algo:clvg1:p} and
\ref{algo:clvg1:n} have to be rerun,
and the new $n^s$, $s$ and flag need
to be calculated.
However, this algorithm completely ignores
that fracture across a grain boundary
is much more complex.

\subsection{Crossing grain boundary}

For a review of the state-of-art
at the beginning of the centry refer, e.g.
to \cite{gerberich2003v8}, specifically
chapters \cite{gerberich2003},
\cite{gerberich2003a}, \cite{milligan2003} and in particular
\cite{tan2003}.
A more fundamental and older account
can be found in \cite{liebowitz1968},
specifically chapters \cite{thomson1968},
\cite{bilby1968} and particularly \cite{beachem1968}.

One major problem of a cellular
approach is that this is a local
approach.
It is very hard to model planes or
other global geometrical entities.
Hence the models of the type used
in \cite{smith2012} can deal with crossing
of the grain boundary easier.
In such geometrical (global) models
the process is simple: as soon as a
cleavage crack reaches a grain boundary
at some spatial point, a cleavage plane
in the following grain is fully determined,
thus allowing for the analysis of the
fracture of the boundary fragment
defined by the grain boundary plane
and by the two cleavage planes in both
grains.
In contrast, in a cellular (local) model,
there is no global cleavage plane defined
in a grain.
When each new cell is analysed, it has
a chance of propagating cleavage into
a neighbouring grain, and hence starting
another cleavage plane.
If left unchecked, this process quickly
leads to the bulk of the model being
populated with cleaved cells.
Such model is, of course, not physical.

Note that the geometrical model is not
physical either.
It simply looks at the final result of
the cleavage propagation and tries to
reproduce it.
It is doubtful that the physical reality
of cleavage propagation is close to
the global geometrical view.

Anyway, in a cellular model some extra
criteria has to be added to prevent
proliferation of new cleavage cracks
at the boundary.
We introduce another logical array,
which records the status of crack
crossing the boundary between any
two grains.
In an undamaged model, this array
is \texttt{.FALSE.} everywhere.
When the cleavage crack crosses a grain
boundary between grains A and B,
then the corresponding entry in this
array
is changed to \texttt{.TRUE.}.

In addition, a new failure state is
added: \texttt{state\_gb\_failed}.

Cells on the grain boundary are cells that
have neighbours from more than 1 grain,
e.g. grains A and B.
Such cells can fail if (a) the crack propagation
status between A and B is \texttt{.FALSE.}
or (b)
if the crack propagation status is \texttt{.TRUE.}
and the central cell has a neighbour of the type
\texttt{state\_gb\_failed}.
The first case corresponds to the cleavage
crack propagating from one grain to another by
crossing the grain boundary.
The second case corresponds to the failure of the
grain boundary itself.
This algorithm is surely artificial, however,
I don't know how the actual physical process
takes place.

In this algorithm there is only one cleavage
plane generated in a grain from a particular
boundary.
However, it is still possible to have multiple
cleavage cracks in a grain if another cleavage
crack starts to propagage from another boundary.
