\section{\texttt{CG\_CLVG}: cleavage propagation
through CA array}

\label{sec:cg_clvg}

\subsection{Purpose}

The aim of this routine is to propagate a cleavage
crack through the CA array.

\subsection{See also}

\hyperref[sec:cg_cpp]{\texttt{CG\_CPP}} 

\subsection{Interface}

\begin{verbatim}

! IN: ITERS - INTEGER, HOW MANY ITERATIONS OF FRACTURE PROPAGATION TO DO
!       SP1 - REAL, MAX PRINCIPAL STRESS VALUE
!       SPV - VECTOR IN THE DIRECTION OF THE MAX PRINCIPAL STRESS
!   SLFSMLR - IF .TRUE., ENFORCE SELF-SIMILAR BOUNDARY CONDITIONS
!     DEBUG - LOGICAL. IF .TRUE., WILL DUMP *LOTS* OF DEBUG OUTPUT
! OUT: ERRSTAT - 0 IF SUCCESS, NON-ZERO IF A PROBLEM
!  
!**********************************************************************73
!  
! ONELINE: TRANSGRANULAR CLEAVAGE PROPAGATION
!     
!**********************************************************************73
 
 SUBROUTINE CG_CLVG(ITERS,SP1,SPV,SLFSMLR,DEBUG,ERRSTAT)

 USE CG_MOD
 IMPLICIT NONE

 INTEGER,INTENT(IN) :: ITERS
 REAL(KIND=8),INTENT(IN) :: SP1,SPV(3)
 LOGICAL,INTENT(IN) :: SLFSMLR,DEBUG
 INTEGER,INTENT(OUT) :: ERRSTAT

\end{verbatim}

\subsection{Details}

We base our ideas on fracture on books like \cite{averbach1959}

The basic ideas are:

\begin{itemize}

\item

Just as with solidification, scan through alive (un-fractured cells).

\item

A crack will grow when alive cell will acquire state ``fractured''.

\item

There must be a distinction between crack front, which can grow,
and crack flanks, which can't grow.

\item

The process must be probabilistic, with the probability of cleavage
propagating depending on the direction of the maximum principal
stress, and the location within the crack (front vs flanks).

\end{itemize}

These ideas lead to the cleavage modelling flowchart, shown
in Fig. \ref{fig:cl:fc}

\begin{figure}[htb]
\centering
\includegraphics[height=0.9\textheight]{./cleavage-fc.pdf}
\caption{Flowchart of cleavage modelling in
\hyperref[sec:cg_clvg]{\texttt{CG\_CLVG}} 
.}
\label{fig:cl:fc}
\end{figure}

We now describe the flowchart from top to bottom.

Scan through all cells in a CA array. $i$ is the index of the
current cell. At \texttt{START} of each iteration, increment $i$.
If $i>N$, where $N$ is the total number of cells in the array,
then there are no more cells, so this is the \texttt{END} of the process. 

If $i\le N$, then make sure cell $i$ is alive (remember we scan
through alive cells!). If not, then get back to the beginning
and start next iteration, increment $i$.

If cell $i$ is alive,
then make sure it's at the crack flank.
We use a number of neighbouring failed cells as the criterion
for this. The idea is that alive cell touching a crack front
will have few failed cells. When alive cell touches crack
flank, it will have lots of failed cells. At present we set
the threshold at 8 failed cells. This is based on a $3\times3=9$ cell
square - one plane of a $3\times 3 \times 3$ neighbourhood.
If alive cell is touching a crack flank, it is likely to 
have $\geq 9$ failed neighbours.

If cell $i$ is not at crack tip, go to the beginning, increment
$i$, and start next iteration.

Randomly choose a failed cell $j$ from the neighbourhood of cell $i$.
(Note that in solidification we did not worry about a specific
state of a neighbouring cell. However, the physical origins of
cleavage lead to the idea that, once started to propagate, a fast running
cleavage crack is extremely likely to continue to run, until
the catastrophic fracture results, or until the crack driving
force falls dramatically. This means we don't want to allow
copying of ``alive'' to ``alive'', because this will slow
down the propagation rate and will make modelling of cleavage
propagation at real time speeds hard.)

Once a failed cell $j$ has been identified, calculate
the unit vector from $i$ to $j$, let's call it $\vec{a}$.

Then do a dot product of $\vec{a}$ and the unit vector
showing the direction of the maximum principal stress,
$\vec{n}$. The dot product is 1 if the vectors are parallel,
and 0 if they are normal. This means we can use
$1-\texttt{dot product}(a,n)$ as a probability of cleavage.
The probability is high if the dot product is close
to 0, and low if the dot product is close to 1.

The curve is shown in Fig. \ref{fig:cl:prob}

\begin{figure}[htb]
\centering
\includegraphics[width=1.3\textwidth]{./cleavage1.pdf}
\caption{Probability of cleavage as a function of the
dot product of the normal vector and the direction vector for
the chosen cell, $n\cdot a$. Factor $N$ is varied from
1 to 10, with the highest $N$ giving the highest gradient.}
\label{fig:cl:prob}
\end{figure}

We take another random number, \texttt{RND}, in the interval [0\ldots 1).
Finally we check that  $1-\texttt{dot product}(a,n) < \texttt{RND}$.
If true, then cell $i$ becomes ``failed''.
Looking at Fig. \ref{fig:cl:prob} an angle of 80$^\circ$
gives $>80$\% probability of cleavage; an angle of 60$^\circ$
degrees gives about 50\% probability of cleavage; and for
an angle of 40$^\circ$ the probability of cleavage
drops to about 20\%. Of course, the shape of this curve,
and therefore the probabilities, can be changed in future.

Then new iteration starts.
